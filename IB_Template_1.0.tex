\documentclass{article}[12]
\usepackage{graphics}
\usepackage{amsmath}
\usepackage[top=4cm, bottom=4cm, left=3.5cm, right=3.5cm]{geometry}
\usepackage{moresize}
\usepackage{anyfontsize}
\usepackage{makeidx}
\usepackage{mathptmx}
\usepackage{titlesec}
\usepackage{setspace}
\usepackage{indentfirst}
\global\parskip 6pt
\titleformat{\title}
	{\normalfont\fontsize{18}{15}\bfseries}{\thetitle}{1em}{}
\titleformat{\section}
	{\normalfont\fontsize{18}{15}\bfseries}{\thesection}{1em}{}
\titleformat{\subsection}
	{\normalfont\fontsize{14}{15}\bfseries}{\thesubsection}{1em}{}
\doublespacing
\begin{document}
\begin{titlepage}
	\begin{center}
		\topskip0pt
		\vspace*{\fill}
		\Huge{This is a Sample Title} %Swap in your title within the brackets
		\vskip .10in
		\normalsize{Author's Name} %Swap in your name within the brackets
		\date{today}
		\vspace*{\fill}
	\end{center}
\end{titlepage}
\begin{center}
	\topskip0pt
	\vspace*{\fill}
	\tableofcontents
	\vspace*{\fill}
\end{center}
\newpage
\section{Introduction}
%To create a new section, type \section{Name of Section}
%To create a new subsection, type \subsection{Name of Subsection}
\section{Conclusion}
\begin{thebibliography}{100}
	%To cite a source, type \bibitem{citekey} followed by your source. The citekey is the argument of \cite, which you should type wherever you want your cited source to appear in your text. See examples for clarification
\end{thebibliography}
\end{document}